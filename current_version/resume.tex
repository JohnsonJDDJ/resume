\documentclass{two-column-resume}
\usepackage[hidelinks]{hyperref}
\usepackage{enumitem}
\usepackage{geometry}
\geometry{top=0.8cm, left=1cm, right=1cm}

\begin{document}
\name{Zhihao (Johnson)}{Du}
\contact{\href{mailto:zhihao617@berkeley.edu}{zhihao617@berkeley.edu} | +1 (510) 833-4417}
\headerline
\noindent
\begin{minipage}[t]{0.27\textwidth}
    \vspace{0.25\baselineskip}
    \section{Links}
        \shortitem{Github}
        \itemdetails{\href{https://github.com/JohnsonJDDJ}{github.com/JohnsonJDDJ}}
        \shortitem{Personal Website}
        \itemdetails{\href{http://zhihao.myxd.place}{zhihao.myxd.place \\ (for info on all projects)}}
    \section{Education}
        \longitem{Bachelor of Arts}{Statistics}{UCBerkeley}{05/2023}
        \longitem{Bachelor of Arts}{Computer Science}{UCBerkeley}{05/2023}
    \section{Skills}
        \shortitem{Languages}
        \itemdetails{
            Python (since 2017)\\%
            SQL (since 2019)\\
            Java (since 2020)\\
            R (since 2020)\\%
            C (since 2021)\\%
            HTML, CSS (since 2017)\\%
        }
        \shortitem{Coursework}
        \itemdetails{
            Machine Learning\\%
            Database Systems\\%
            Reproducible Data Science\\%
            General Linear Models\\%
            Data Structures\\%
            Machine Structures\\%
            Linear Algebra\\%
        }
        \shortitem{Tools}
        \itemdetails{
            Git (since 2020)\\%
            Jupyter (since 2020)\\%
            Microsoft Azure (since 2022)\\%
            DBeaver (Summer 2021)\\%
            Command Line\\%
            Microsoft Office\\%
        }
\end{minipage}% This hacky percent sign is for removing the space between two columns.
\begin{minipage}[t]{0.73\textwidth}
    \vspace{0.25\baselineskip}
    \section{Experience}
        \longitem{ETL Engineer Intern}{}{DataCVG Shanghai}{05/2021 - 08/2021}
        \itemdetails{
        Helped the client "FosunPharma" manage their database system with pharmaceutical data by performing extract-transform-load (ETL) on relational databases. The project involved the following processes:
        \vspace{-0.5\baselineskip}
            \begin{itemize}[itemsep=-2.5pt]
                \item Designed the architecture for the target relational database.
                \item Wrote queries to combine tables from two relational data sources.
                \item Debugged architecture failures through long diagnostic process.
            \end{itemize}
        }
    \section{Research}
        \longitem{Python ML Engineer}{}{Project AEI - Koer A.I., Inc.}{01/2022 - Now}
        \itemdetails{
            Responsible for the algorithm behind emotion discernment and early warning system for police aggression. Built a parallel CNN transformer using pytorch to discern emotion from human voice. The steps are:
            \vspace{-0.5\baselineskip}
            \begin{itemize}[itemsep=-2.5pt]
                \item Trained the model using large emotional databases (RAVDESS, SAVEE).
                \item Preprocessed data through augmentation using librosa and pytorch.
                \item Training process carried through Azure ML cloud compute platform.
            \end{itemize}
        }
    \section{Projects}
    		\longitem{ZileanLeague}{}{Machine Learning in Python}{06/2022 - Now}
        \itemdetails{
            Predicted League of Legends match outcomes with game statistics before the 16 minute mark. Tuned a random forest classifier and a XGBoost classifier.
        }
        \longitem{zilean}{}{Python Package Development}{05/2022 - Now}
        \itemdetails{
            Developed a python package "zilean" that can extract and process complex objects from the Riot Games API. It essentially builds a bridge between the API and the traditional python data science environment. Installable via pip, with full documentation and test coverage.
        }
        \longitem{Forest Fire Prediction}{}{Reproducible Data Science in Python}{04/2022 - 05/2022}
        \itemdetails{
            A forest fire prediction project with a robust and consistent reproduciblility framework. Tools and technologies include: Makefile, myBinder, Jupyter Book, Github Pages, Github Actions, and unit tests for an installable python package.
        }
        \longitem{HOYO Lab}{}{Data Modelling in R}{03/2021 - 08/2021}
        \itemdetails{
            Predicted Genshin Impact's damage calculation algorithm by building a basic linear model, then displayed the result using R ShinyApp. Data collected from in-game simulations. Used the model to wrote game tutorials receiving 400,000+ views.
        }
\end{minipage}
\end{document}
